%\pagestyle{fancy}
\chapter*{Kurzfassung}
\label{ch:Kurzfassung}
\gls{avp} ist eine vielversprechende Anwendung des autonomen Fahrens, bei der ein Auto selbstst\"andig in einem Parkhaus navigiert.
Die Navigation in einem Parkhaus ist anspruchsvoll, insbesondere Rampen stellen aufgrund der beengten Platzverh\"altnisse ein Problem dar.
In dieser Arbeit wird ein Rampenerkennungssystem entwickelt.
Es werden drei verschiedene Sensoren verwendet und miteinander verglichen.
Eine Inertiale Messeinheit  (engl. \gls{imu}) wird verwendet, um die Steigung der Stra\ss e zu messen und anhand dessen festzustellen, ob sich das Fahrzeug auf einer Rampe befindet.
Es werden verschiedene Methoden getestet, darunter eine, bei der die Messungen des Raddrehzahlsensors mit den \gls{imu}-Daten fusioniert werden.
Um festzustellen, ob sich das Fahrzeug einer Rampe n\"ahert, werden ein \gls{lidar}-Sensor und eine Kamera verwendet.
Anhand der \gls{lidar}-Daten kann der Abstand zur Rampe und die Eigenschaften der Rampe, wie Winkel, Breite und L\"ange, bestimmt werden.
Die Erkennung mit der Kamera basiert auf einem neuralen Netzwerk, dem Mask \gls{rcnn}.
F\"ur das Training des Netzwerks wird ein eigener Datensatz erstellt.
Anhand der Daten von echten Testfahrten werden die Ergebnisse der verschiedenen Methoden miteinander verglichen.
Die Rampenerkennung mit dem \gls{lidar} und der Kamera funktionierte sehr gut, aber die Ergebnisse unter Verwendung der \gls{imu} waren aufgrund der Qualit\"at des Sensors beschr\"ankt.