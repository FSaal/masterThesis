%\pagestyle{fancy}
\chapter*{Abstract}
\label{ch:Abstract}
\gls{avp} is a promising application of autonomous driving, in which a car navigates on its own in a parking garage.
Navigating in parking garages is challenging, ramps in particular pose a problem due to the compact space.
In this thesis a ramp detection system is developed.
Three different sensors are used and compared to each other.
An \gls{imu} is used to measure the road grade and determine from it if the car is on a ramp.
Different methods are tested, including one where the wheel speed sensor measurements are fused with the \gls{imu} data.
To determine if the car is approaching a ramp, a \gls{lidar} sensor and a camera are used.
Using the \gls{lidar} data, the distance to the ramp and properties of the ramp, such as the angle, width and length can be determined.
The detection using the camera is based on the Mask \gls{rcnn}.
An own dataset is created for the training of the network.
The performance and accuracy of the different sensors and methods are evaluated in a real life scenario.
The detection using the \gls{lidar} and camera worked very well, but the results using the \gls{imu} were limited by the quality of the sensor.
