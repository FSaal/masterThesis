\chapter{Conclusion}
\label{ch:Conclusion}
% Do not write long version of imu or lidar acronym again
\glslocalunset{avp}
\glslocalunset{coco}
\glslocalunset{map}
\glslocalunset{imu}
\glslocalunset{lidar}
\glslocalunset{ransac}
\glslocalunset{rcnn}

\section{Summary}
The detection of ramps in parking garages for \gls{avp} is a problem...
\itodo{One sentence about why this thesis was done}
In this thesis different sensors and methods for the detection and classification of ramps in parking garages were developed and evaluated.
Using an \gls{imu}, a ramp could be detected by measuring the tilt of the car.
Different online methods were evaluated, the best results were achieved using a complementary filter, which fuses the accelerometer data and gyroscope data of the \gls{imu}.
By also using the data from the wheel speed sensors, an accurate estimation of the length of the ramp was possible, in addition to the average angle estimation of the ramp.

To be able to detect a ramp before entering it, different sensors were used.
At first a \gls{lidar} was tested.
Using the point cloud generated by the \gls{lidar} the planar ramp region could be extracted using the \gls{ransac} algorithm.
The length and width of the ramp could be measured, as well as the average angle of the ramp and the distance of the car to the ramp.
The ramp was successively detected in more than 95\% of the cases, if the distance to the ramp was less than \SI{25}{\metre}.
For both the \gls{imu} and \gls{lidar} a novel calibration method was implemented, which automatically transforms the sensor measurements into the coordinate system of the car.

Lastly, a camera was used to detect the ramp.
The object detection was done using the deep learning Mask \gls{rcnn}, which generates a segmentation mask of the object.
An own dataset was created for the training.
Due to the small size, augmented images were added to the dataset and an on the \gls{coco} dataset pretrained network was used.
The network proofed to be able to detect ramps well and achieved a \gls{map} score of over 90\%.

Lastly the predicted segmentation mask was elevated into 3D space, by projection a 3D point cloud onto an image, removing points outside the mask, and then transforming them back into 3D space.
This was done for the point cloud generated by a stereo-camera, and the point cloud of the \gls{lidar}.
This allowed for the accurate detection of the drivable part of the ramp in 3D space, whereas the \gls{lidar} algorithm did not make a distinction between the drivable part and the curb sides.



\section{Conclusion}
The \gls{imu} methods were suspect.
Due to the significant drift of the gyroscope, the best results were achieved by using a complementary filter, but using the gravity method, also good results were achieved.

Since the \gls{lidar} used in the experiments is intended to be used as a 360 top \gls{lidar}, the vertical resolution was limited.
This leads to length and distance fail and could be solved by mems lid?



\section{Future Work}
In this thesis different sensors were used to detect a ramp and measure its properties.
While relatively good results could be achieved, each method could be improved further.
For the \gls{imu} other algorithms such as a Kalman filter could be tested.
Furthermore, the problem of the delay between the wheel speed sensor and \gls{imu} measurements could be investigated.
The same goes for the \gls{lidar}, other algorithms than \gls{ransac} could be tested.
There exist newer algorithms which are more efficient, which means that the downsampling and passthrough filtering could be reduced.
The neural network for the object detection could also be made more robust by using more data, preferably of other environments, in the training.

To further improve the results, the different methods could also be combined.
In the thesis the online implementation of the objection detection was not carried out, this could be done in the future.
Then, the \gls{lidar} point cloud extraction using the instance segmentation prediction could also be properly evaluated.
Another option of sensor fusion would be to use the estimated ramp properties using the \gls{lidar} data and correct them after driving on the ramp using the \gls{imu} data.

For the evaluation of the different methods the \gls{lidar} data was used, which is subject to error.
Additionally, the testing in a more controlled environment would be useful.
This could be done in a simulated environment, e.g. in \texttt{CARLA}~\footnote{\url{https://github.com/carla-simulator/carla}}~\cite{Dosovitskiy2017}.

Another potential future task would be to use the implemented methods to create a meta map of the environment.
If a map is created, e.g. using the \gls{lidar} and the \texttt{hdl\_slam} library, the location of the ramp could be labeled in the map, together with the properties of the ramp.



\section{Advantages and disadvantages of different methods}
\iimprov{Just a really short sketch of what could be written here}
In this section the different sensors and methods will be compared.
At first the estimation using an \gls{imu} was tested.
An advantage is that almost all modern cars are already equipped with an \gls{imu} and wheel encoders.
\dots disadvantage?

To detect a ramp before entering it, a \gls{lidar} and a camera were used.
Using a \gls{lidar}, a very accurate measurement estimation of the width and angle of the ramp was possible.
But the distance estimation and length estimation was limited by the vertical resolution.
The \gls{lidar}
While good results were achieved in the one garage, other garages?
...
Furthermore, \glspl{lidar} are only available in premium vehicles SOURCE due to its high price, but with the ... in \gls{mems} \glspl{lidar} this could change in the future.


A camera on the other hand is fairy cheap and in most cars.


Using the \gls{imu} and odometer, no additional hardware is needed, since both sensors already available in most currently available cars.
The \gls{lidar} provides great accuracy and range, but is not used in many cars due to its price.
But this could change in the future, when \gls{mems} \glspl{lidar} become cheaper.
The camera is available in most cars, but for a robust neural network, a bigger dataset is needed.
