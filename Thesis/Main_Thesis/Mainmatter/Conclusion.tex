\chapter{Conclusion}
\label{ch:Conclusion}
% Do not write long version of imu or lidar acronym again
\glslocalunset{coco}
\glslocalunset{map}
\glslocalunset{imu}
\glslocalunset{lidar}
\glslocalunset{ransac}
\glslocalunset{rcnn}

\section{Summary}
The detcetion of ramps in parking garages for \gls{avp} is a problem...
In this thesis different sensors and methods for the detection and classification of ramps in parking garages were developed and evaluated.
Using an \gls{imu}, a ramp could be detected by measuring the tilt of the car.
Different online methods were evaluated, the best results were achieved using a complementary filter, which fuses the accelerometer data and gyroscope data of the \gls{imu}.
By also using the data from the wheel speed sensors, an accurate estimation of the length of the ramp was possible, in addition to the angle estimation.

To be able to detect a ramp before entering it, different sensors were used.
At first a \gls{lidar} was tested.
Using the point cloud generated by the \gls{lidar} the planar ramp region could be extracted using the \gls{ransac} algorithm.
The length and width of the ramp could be measured, as well as the average angle of the ramp and the distance of the car to the ramp.
Since the \gls{lidar} used in the experiments is intended to be used as a 360 top \gls{lidar}, the vertical resolution was limited.
This leads to length and distance fail and could be solved by mems lid?

For both the \gls{imu} and \gls{lidar} a novel calibration method was implemented, which automatically transforms the sensor measurements into the coordinate system of the car.

Lastly, a camera was used to detect the ramp.
The object dectection was done using a deep learning approach, the Mask \gls{rcnn} network was trained to detect ramps in images.
An own dataset was created, but using an on the \gls{coco} dataset pretrained network allowed for short training times and a more accurate result.
The network proofed to be able to detect ramps well and achieved a \gls{map} score of over 90\%.
Lastly the predicted segmentation mask was elevated into 3D space, by projection a 3D point cloud onto an image, removing points outside the mask, and then transforming them back into 3D space.
This was done for the point cloud by the ZED camera, which uses stereo-vision to generate a point cloud, and the point cloud of the \gls{lidar}.
This allows for the accurate detection of the drivable part of the ramp, whereas the \gls{lidar} algorithm did not make a distinction between the drivable part and the curb sides.



\section{Conclusion}
The \gls{imu} methods were suspect .
Due to the significant drift of the gyroscope, the best results were achieved by using a complementary filter, but using the gravity method, also good results were achieved.

The



\section{Future Work}
In this thesis different sensors were used to detect a ramp and measure its properties.
Each method could be improved further.
For the \gls{imu} other algorithms such as Kalman filter could be tested.
The same goes for the \gls{lidar} algorithm which is based on the \gls{ransac} algorithm.
There exist newer algorithms which are even more efficient, which would allow for the use of a point cloud of higher resolution.
The neural network could be made more robust by using more data, preferably of other environments, for the training.

To further improve the results, the different methods could also be combined.
As described in \cref{ssec:point_cloud_extraction}, the instance segmentation prediction made using the camera image can also be applied to the point cloud.
In the ... of this thesis, an online implementation of this method was not tested.
Another option would be to use the estimated ramp properties using the \gls{lidar} data and correct them after driving on the ramp using the \gls{imu} data.

For the evaluation the \gls{lidar} data was used.
A better .. would be the use of a more controlled environment.
An easy would be the use of a simulated environment, e.g. Carla.

\begin{itemize}
    \item Sensors could be fused
    \item Other \gls{imu} methods such as kalman filter
    \item use camera and extract point cloud from \gls{lidar}
    \item use \gls{lidar} to detect ramp and use it to label ramp in image
    \item octomap
    \item carla
\end{itemize}



\section{Advantages and disadvantages of different methods}
\iimprov{Just a really short sketch of what could be written here}
In this section the different sensors and methods will be compared.
At first the estimation using an \gls{imu} was tested.
An advantage is that almost all modern cars are already equipped with an \gls{imu} and wheel encoders.
\dots disadvantage?

To detect a ramp before entering it, a \gls{lidar} and a camera were used.
Using a \gls{lidar}, a very accurate measurement estimation of the width and angle of the ramp was possible.
But the distance estimation and length estimation was limited by the vertical resolution.
The \gls{lidar}
While good results were achieved in the one garage, other garages?
...
Furthermore, \glspl{lidar} are only available in premium vehicles SOURCE due to its high price, but with the ... in \gls{mems} \glspl{lidar} this could change in the future.


A camera on the other hand is fairy cheap and in most cars.


Using the \gls{imu} and odometer, no additional hardware is needed, since both sensors already available in most currently available cars.
The \gls{lidar} provides great accuracy and range, but is not used in many cars due to its price.
But this could change in the future, when \gls{mems} \glspl{lidar} become cheaper.
The camera is available in most cars, but for a robust neural network, a bigger dataset is needed.
