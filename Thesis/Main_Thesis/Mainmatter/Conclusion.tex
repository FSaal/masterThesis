\chapter{Conclusion}
\label{ch:Conclusion}
\section{Advantages and disadvantages of different methods}
\iimprov{Just a really short sketch of what could be written here}
Using the \gls{imu} and odometer, no additional hardware is needed, since both sensors already avaiable in most currently available cars.
The \gls{lidar} provides great accuracy and range, but is not used in many cars due to its price.
But this could change in the future, when \gls{mems} \glspl{lidar} become cheaper.
The camera is available in most cars, but for a robust neural network, a bigger dataset is needed.

\section{Outlook}
In this thesis different sensors were used to detect a ramp and meassure its properties.
Each method could be improved further.
For the \gls{imu} other algorithms such as Kalman filter could be tested.
The same goes for the \gls{lidar} algorithm which is based on the \gls{ransac} algorithm.
There exist newer algorithms which are even more efficient, which would allow for the use of a point cloud of higher resolution.
The neural network could be made more robust by using more data, preferably of other environments, for the training.\par
To further improve the results, the different methods could also be combined.
For example the \gls{lidar} or camera with the trained network could be used to detect the ramp before entering it.

\begin{itemize}
	\item Sensors could be fused
	\item Other \gls{imu} methods such as kalman filter
	\item use camera and extract point cloud from \gls{lidar}
	\item use \gls{lidar} to detect ramp and use it to label ramp in image
	\item octomap
	\item
\end{itemize}


\section{Conclusion}
In this thesis different sensors and methods for the detection and classification of ramps were developed and evaluated.
Using an \gls{imu}, a ramp could be detected by measuring the tilt of the car.
Different online methods were evaluated, the best results were achieved using a complementary filter, which fuses the accelerometer data and gyroscope data of the \gls{imu}.
Furthermore, the average angle and length of the ramp could be measured.
The most accurate length measurement was achieved when a wheel speed sensor was used, but using an accelerometer when taking ? into accout was proven to also be viable alternative.\par
To be able to detect a ramp before entering two further methods were used.
Using the point cloud generated by the \gls{lidar} the planar ramp region could be extracted using the \gls{ransac} algorithm.
The length and width of the ramp could be measured, as well as the average angle of the ramp and the distance of the car to the ramp.
For both the \gls{imu} and \gls{lidar} a novel calibration method was implemented, which automatically transforms the sensor measurements into the coordinate system of the car.\par
Lastly the validity of the use of a camera was evaluated.
Using a machine learning approach, the Mask-\gls{rcnn} network was trained to detect ramps in images.
An own dataset was created, but using an on the \gls{coco} dataset pretrained network allowed for short training times and a more accurate result.
The network proofed to be able to detect ramps well and achieved a \gls{map} score of over 90\%.
Lastly the predicted segmentation mask was elevated into 3D space, by projection a 3D point cloud onto an image, removing points outside of the mask, and then transforming them back into 3D space.
This was done for the point cloud by the ZED camera, which uses stereo-vision to generate a point cloud, and the point cloud of the \gls{lidar}.The
