%\pagestyle{fancy}
\chapter{Introduction}
\label{ch:Introduction}
\itodo{Write introduction}

\section{Motivation}
\iimprov{Just copied this from the expose, needs improvement\\
    maybe some more sections about the specific topics or about goals}
Autonomous driving \dots

Parking is one of the most challenging driving tasks and the cause of almost half of the car accidents~\cite{accident}.
Current cars are already able to fully automated park on their own in parallel or perpendicular parking spaces.
But due to the very limited space in cities, parking garages are often used in central areas~\cite{Banzhaf2017}.
\gls{avp} allows for a fully automated parking experience.
The car is left in a drop-off zone and finds a parking spot on its own.
Afterwards the driver can give a command and the car leaves the parking spot again and picks up the driver.
AVP saves time, the hassle of remembering the parking level and spot and furthermore allows to use the available space more efficiently and also minimizes the risk of collisions.
For this to work an exact mapping of the environment and localization of the car in the garage is necessary.\\
This can be done either by \gls{slam} or the area can be mapped beforehand (e.g. using \gls{lidar} sensors) in which case only a localization of the car is necessary.
The mapping can be done in 2D or 3D. 2D maps only show information of the current level. Hence if the car is driven up or down a ramp, the new map of the corresponding floor has to be loaded.
Because the localization usually only works in a 2D-plane, a change of levels would not be detected.
To solve this problem, a ramp detection has to be implemented.



\section{Goals}
The goal of this thesis is the development of a ramp detection system.
The system should be able to detect the presence of a ramp and the position of the ramp.
Furthermore, ramp properties such as the slope and the length of the ramp should be estimated.
Different sensors will be used and compared to each other.




\section{Outline}
This thesis is structured in the following way.
In \cref{ch:Background} the theoretical foundations needed for the methods used in this thesis are described.
A brief mathematical overview is followed by a description and explanation of the working principle of the different sensors used in this thesis.
Lastly, a short introduction into the field of computer vision and especially in neural networks is given.
Other existing methods and approaches to solve the problem posed in this thesis are reviewed in \cref{ch:StateOfTheArt}.
The different implemented methods are described in \cref{ch:Methods}.
Different approaches to estimate the road grade using an \gls{imu} are presented in \cref{sec:methods_imu}, one method using a \gls{lidar} in \cref{sec:methods_lidar} and finally using a camera in \cref{sec:methods_camera}.
The exact type of sensors used for the recording of the data and the environment in which they were taken, is described in \cref{ch:ExperimentalSetup}.
\Cref{ch:Results} contains the results of the different methods.
A summary of the thesis and potential future work is given in \cref{ch:Conclusion}.
