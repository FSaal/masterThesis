%\pagestyle{fancy}
\chapter{Introduction}
\label{ch:Introduction}
Note explanation
\itodo{This is TODO note by my self}
\iimprov{This is an improvement note}
\iquest{This is a question I have}
\isug{This is a suggestion from others}

\section{Motivation}
\iimprov{Just copied this from the expose, needs improvement\\
    maybe some more sections about the specific topics or about goals}
Parking is one of the most challenging driving tasks and the cause of almost half of the car accidents \cite{accident}.
Current cars are already able to fully automated park on their own in parallel or perpendicular parking spaces.
But due to the very limited space in cities, parking garages are often used in central areas \cite{Banzhaf2017}.
\gls{avp} allows for a fully automated parking experience.
The car is left in a drop-off zone and finds a parking spot on its own.
Afterwards the driver can give a command and the car leaves the parking spot again and picks up the driver.
AVP saves time, the hassle of remembering the parking level and spot and furthermore allows to use the available space more efficiently and also minimizes the risk of collisions.
For this to work an exact mapping of the environment and localization of the car in the garage is necessary.\\
This can be done either by \gls{slam} or the area can be mapped beforehand (e.g. using \gls{lidar} sensors) in which case only a localization of the car is necessary.
The mapping can be done in 2D or 3D. 2D maps only show information of the current level. Hence if the car is driven up or down a ramp, the new map of the corresponding floor has to be loaded.
Because the localization usually only works in a 2D-plane, a change of levels would not be detected.
To solve this problem, a ramp detection has to be implemented.



\section{Outline}
\todoin{Brief overview over structure of thesis}