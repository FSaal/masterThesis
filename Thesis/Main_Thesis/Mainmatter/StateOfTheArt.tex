%\pagestyle{fancy}
\chapter{State of the art}
\label{ch:StateOfTheArt}

\itodo{Some text}



\section{IMU}
In \cite{Jauch2018} different methods to estimate the road grade angle are discussed.
\itodo{Not really IMU, maybe in prev section instead}
There exist methods without Inertial Sensors relying on a model describing the longitudinal movement of the vehicle and the topology of the road.
Both models are fused using a Kalman filter to improve the accuracy of the estimation \cite{SAHLHOLM200755}.
A Kalman filter is also used in \cite{Sahlholm2010}, where vehicle sensor data and \gls{gps} data are fused.
Besides the road grade, the vehicle mass is often also unknown and estimated as well \cite{Sahlholm2010,Maleej2015}.
Another method using GPS data and IMUs to calculate the vertical and horizontal velocity change respectively and thereby the road grade is proposed in \cite{Ryu2004}.
\cite{YazdaniBoroujeni2014} omits the IMU and relies on a GPS sensor and a barometer.
THe problem with most of these methods are the use of a GPS sensor, which is not precise and most importantly will not work in parking garages.
\itodo{source and why}
Furthermore many internal measurements such as the engine torque, brake system usage, selected gear etc. can not easily be accessed and thus might not be available.\\
A method which does not use GPS, but only accelerometers and wheel odometers instead is described in \cite{Nilsson2012}.
The vehicle acceleration, calculated by deriving the wheel speed measurements in respect to time, is subtracted from the accelerometer signal in longitudinal direction.
The remaining part is then the gravitational acceleration and can be used to calculate the road grade angle.
A similar approach is used in \cite{Sentouh2008}.
\itodo{Maybe first write about acceleration only method, describe its problems, and then the improve methods such as complementary filter (adds gyro)}
\todo{Read} \cite{He2020} apparently even better than complementary filter or acceleration method.
More about \cite{Palella2016} (is actually used for the implementation by me).
\dots\\
\itodo{Now that foundation of available is briefly explained, write something more specific about the methods which will be used. Or does this go in another section?}



\section{LiDAR}
\todoin{\begin{itemize}
    \item Lidar methods for plane detection
    \item If found, methods for elevation estimation
\end{itemize}
}



\section{Camera}
\todoin{\begin{itemize}
    \item Something simple
    \item Stereo vs mono
\end{itemize}
}