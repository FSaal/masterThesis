%\pagestyle{fancy}
\chapter{State of the art}
\label{ch:StateOfTheArt}
\itodo{Add short overview}
\isug{Move everything from this chapter to the corresponding subsection in chapter 3 (e.g. \gls{imu} state of the art after \gls{imu} background)}
\isug{Or just switch this chapter with Background chapter}
...
In this chapter an overview of different existing methods to detect ramps or measure the road grade is given.
It is divided into three sections, each focusing on work which relies mainly on the \gls{imu}, \gls{lidar} or camera sensor.

\section{\glsentryshort{imu}}
\iimprov{Section also contains other methods without imu}
In ref.~\cite{Jauch2018} different methods to estimate the road grade angle are discussed.
There exist methods without Inertial Sensors relying on a model describing the longitudinal movement of the vehicle and the topology of the road.
Both models are fused using a Kalman filter to improve the accuracy of the estimation \cite{Sahlholm2007}.
A Kalman filter is also used in ref.~\cite{Sahlholm2010}, where vehicle sensor data and \gls{gps} data are fused.
Besides the road grade, the vehicle mass is often also unknown and estimated as well, using common sensors of heavy-duty vehicles \cite{Sahlholm2010, Maleej2014}.
More methods such as recursive least squares, extended Kalman filtering and a dynamic grade observer are discussed in ref.~\cite{Kidambi2014}
Another method using \gls{gps} data and \glspl{imu} to calculate the vertical and horizontal velocity change respectively and thereby the road grade is proposed in ref.~\cite{Ryu2004}.
\cite{YazdaniBoroujeni2014} omits the \gls{imu} and relies on a \gls{gps} sensor and a barometer.\\
\gls{gps} satellites broadcast information about their position and exact time to a \gls{gps} receiver, which than can calculate its position using triangulation \cite{Mainetti2014}.
While an accuracy of up to \SI{1}{\metre} can be achieved when outside, the performance significantly drops when used indoors.
The radio waves sent from the satellites are scattered, attenuated or blocked completely by walls and other obstacles, resulting in a very weak or even a complete loss of the signal \cite{Ozdenizci2015}.\\
Most methods mentioned above do not seem fit for the task, due to the reliance on \gls{gps}.
Furthermore, many internal measurements such as the engine torque, brake system usage, selected gear etc. can not easily be accessed and thus might not be available.\\
A method which does not use \gls{gps}, but only accelerometers and wheel odometers instead is described in ref.~\cite{Nilsson2012} and \cite{Palella2016}.
The vehicle acceleration, calculated by deriving the wheel speed measurements in respect to time, is subtracted from the accelerometer signal in longitudinal direction.
The remaining part is then the gravitational acceleration, which is zero if driving on flat ground but not anymore if driving on an elevated road, and can be used to calculate the road grade angle.
A similar approach is used in ref.~\cite{Sentouh2008}.
\cite{He2020} adds a gyroscope to the accelerometer and fuses their estimations using a quaternion unscented Kalman filter.
The gyroscope measurements get integrated over time to receive the pitch angle.
The angle from the angular velocity is accurate in short-term, but is suspect to drifting over time.
The drift can be corrected by using the accelerometer signal, which is accurate in the long-term, but unlike the gyroscope not accurate in the short-term.
\cite{Wu2016} uses all components of an \gls{imu} (meaning also the magnetometer) and fuses them using a complementary filter.
The estimated quaternions using the accelerometer and angular velocity measurements respectively are fused, and the magnetometer data is used to improve the quaternion estimation from the accelerometer, but only if there are no magnetic disturbances.
\cite{Euston2008,Jauch2018} also use a complementary, but fuse the estimated angle from the accelerometer and gyroscope instead of the quaternions.



\section{\glsentryshort{lidar}}
\itodo{Probably merge \gls{lidar} and camera sections together}
While an \gls{imu} allows for the estimation of the current road grade angle, it can not be used to detect changes in the road grade ahead of the vehicle.
In the case of ramps, the road grade angle of the ramp can be seen as near constant and thus the ramp can be classified as a planar surface.
By also detecting the ground plane, the transition from ground to ramp can be determined.
Several techniques to detect planes in a point cloud exist.
As mentioned in ref. \cite{Gallo2008}, there exist mainly three different approaches to segment and estimate planar regions in a point cloud simultaneously.
The first approach, which will also be used in this master thesis, is to iteratively extract the most dominant plane from the point cloud.
After a plane has been detected, its inliers are removed from the point cloud and the remaining points will be searched for the next biggest plane, until not enough points are left to build a proper plane.
In the second approach, all visible planes are estimated simultaneously.
Either using the Expectation-Maximization algorithm \cite{Liu2001, Triebel2005} or by using the Generalized \gls{pca} algorithm \cite{Vidal2005}.
A disadvantage of this approach is that the number of planar surface elements in the scene must be estimated.
The region growing algorithms \cite{Besl1988, Taubin1991} build the third family of algorithms.
Starting from a certain region, neighboring points are added to the region if they share a similar model.
Different patches can than be merged together, if they are consistent with each other.
It is a fast and simple algorithm, but depends on a good selection of the first starting points.\\
Since the point cloud is usually not free of outliers, a robust procedure must be employed to remove them.
The most well known are the \gls{ransac} \cite{Fischler1981} and Hough transform \cite{Illingworth1988}.

While it does seem that no research exists, where a \gls{lidar} is used to detect ramps, several papers discuss the detection of planes in point cloud data and estimate road grade?
Different methods exist where the \gls{lidar} recordings from a plane are used to map the road grade.
In ref. \cite{El-Sayed2018} a new technique for plane detection is proposed.
The point cloud is down-sampled using octree into small cubes, which are then down-sampled based on their local density.
\gls{pca} is used to find planar surfaces.
It is able to process large point clouds faster than the efficient \gls{ransac} algorithm, while achieving a higher accuracy.
Ref. \cite{Sakenas2007} introduces a new algorithm, to extract planar maps from 3D data.
The point cloud is divided into 3D cells and a histogram over the z-axis is created.
Ramps can then be detected by searching for neighboring cells, where the height increases gradually.
\todoin{
	\begin{itemize}
		\item \gls{lidar} methods for plane detection
		\item If found, methods for elevation estimation
	\end{itemize}
}



\section{Camera}
\todoin{
	\begin{itemize}
		\item Something simple
		\item Stereo vs mono
	\end{itemize}
}
Ref.~\cite{Nejati2016} used an RGB-D sensor (camera image + depth sensor) to detect ramps for wheelchairs.
Ramp properties such as angle, width, length and the orientation of the ramp were determined as well.
A ground plane estimation using a disparity map, which can be generated from a stereo camera, is proposed in \cite{Chumerin2008}.



\section{Outlook?}
Due to the available sensor stack, not all mentioned methods can be tested.
While a Kalman filter achieves very good results, it is generally complex and the precise knowledge of process and measurement noise is necessary \cite{Higgins1975}
The gravity method and complementary filter will be tested for the \gls{imu}.
Because no method to detect ramps using a \gls{lidar} could be found, an own method has to be implemented.
For the plane detection the \gls{ransac} algorithm will be used, since it is already implemented in the PCL library ADD LINK.